\documentclass[../../Thesis.tex]{subfiles}
\begin{document}
\header{Abstract}
Over the last few years, word embeddings have taken a dominant position in the Information Retrieval domain. Many studies have been done concerning the quality of word embeddings on general texts, such as the Wikipedia corpus and comments on review websites. Giving promising results, the word embeddings have been studied and improved over recent years. However, these studies have been focused on generic text, which do not contain the complexity of in-domain knowledge, such as rare domain specific words. This research focusses on the usefulness of word embeddings on domain specific texts, concerning scientific articles which have been published in 2017. The data shows that the word embeddings only perform better than TF-IDF, which was chosen as a traditional alternative, when the ranking is done on titles. The best ranking result is obtained through a TF-IDF with a vocabulary size of 10.000 applied on the abstracts, resulting in a median rank of 14, while the pure word embeddings result in a median rank of 23. This shows that word embeddings, although useful, are not unchallenged by older techniques.
\end{document}